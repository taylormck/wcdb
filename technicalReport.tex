\documentclass[12pt]{report}

\usepackage[pdftex]{graphicx} % for pictures

\usepackage{color}
\usepackage{listings}
\usepackage[usenames,dvipsnames,svgnames,table]{xcolor}

\usepackage{marvosym} % Required for symbols in the colored box
\usepackage{ifsym} % Required for symbols in the colored box

\usepackage{hyperref}
\definecolor{linkcolor}{HTML}{506266} % Blue grey color for links
\hypersetup{colorlinks,breaklinks, urlcolor=linkcolor, linkcolor=linkcolor} % Set up links and colors

\definecolor{shade}{HTML}{F5DD9D} % Peach color for the contact information box
\definecolor{text1}{HTML}{2b2b2b} % Main document font color, off-black
\definecolor{headings}{HTML}{701112} % Dark red color for headings

\usepackage{fancyhdr}
% \pagestyle{fancy}
% \fancyhf{}

\usepackage{titlesec} % Allows creating custom \section's

\titleformat{\section}{\color{headings}
\scshape\Large\raggedright}{}{0em}{}[\color{black}\titlerule]
\titlespacing{\section}{0pt}{0pt}{5pt} % Spacing around titles

\title{Great Moonstone Oxen of the First and Forsaken Oceans}
\author{
    Taylor McKinney
    \and Matt Stringer
    \and Cesar Cantu
    \and Spencer Reynoso
    \and Ben Bowley-Bryant
    \and David Denton}

\begin{document}
\maketitle

\section*{Introduction}
\subsection*{What is the problem?}
It's often easy to forget the many crises occurring throughout the world.
This site is designed to gather the data on every crisis, and present in a simple format.
We gather details about victims, and information on organizations and people that want to help.

\subsection*{What are the use cases?}
With the increasing availability of telecommunication access and this site makes use of this to share news and help victims.
This site can be a good source for anyone looking to get informed and help. 
We can provide contact information for people and organizations involved in helping the victims of crises around the world.
People who want to help can look to our site for the best ways to get involved, whether it be donating to the right charities or volunteering at a local site.

\newpage
\section*{Design}
\subsection*{XML Schema}
Our class has been divided into separate groups for this project, but most groups have agreed upon a shared schema for XML.
The schema as of the latest version is printed here, but the latest version can be found with this link.
\newline
\href{https://github.com/aaronj1335/cs373-wcdb1-schema/blob/master/WorldCrises.xsd.xml}{Updated schema on Github}
\newline

% Showing xml code
\lstset{language=XML,
        breaklines=true,
        frame=shadowbox,
        rulesepcolor=\color{gray},
        xleftmargin=-15pt,
        xrightmargin=-20pt,
        framexleftmargin=15pt,
        tabsize=2, 
        basicstyle=\tiny, 
        numbers=left, 
        numberstyle=\tiny,
        keywordstyle=\color{blue}\bf,
        commentstyle=\color{gray},
        stringstyle=\color{magenta},
        numbersep=5pt,
        emph={
            xsd:schema,
            xsd:element,
            xsd:complexType,
            xsd:sequence,
            xsd:key,
            xsd:selector,
            xsd:field,
            xsd:keyref,
            xsd:attribute,
            xsd:simpleType,
            xsd:restriction,
            xsd:any
        },
        emphstyle={\color{purple}}
        }
\begin{lstlisting}
<?xml version="1.0" ?>
<!-- XSD for World Crises database -->
<xsd:schema xmlns:xsd="http://www.w3.org/2001/XMLSchema">
    <xsd:element name="WorldCrises">
        <xsd:complexType>
            <xsd:sequence>
                <xsd:element name="Crisis" type="CrisisType" minOccurs="1" maxOccurs="unbounded" />
                <xsd:element name="Person" type="PersonType" minOccurs="1" maxOccurs="unbounded" />
                <xsd:element name="Organization" type="OrgType" minOccurs="1" maxOccurs="unbounded" />
            </xsd:sequence>
        </xsd:complexType>
        
        <!-- All keys and keyrefs go here -->
        <xsd:key name="CrisisKey">
            <xsd:selector xpath="Crisis" />
            <xsd:field xpath="@ID" />
        </xsd:key>
        <xsd:keyref name="CrisisKeyRef_Person" refer="CrisisKey">
            <xsd:selector xpath="Person/Crises/Crisis" />
            <xsd:field xpath="@ID" />
        </xsd:keyref>
        <xsd:keyref name="CrisisKeyRef_Org" refer="CrisisKey">
            <xsd:selector xpath="Organization/Crises/Crisis" />
            <xsd:field xpath="@ID" />
        </xsd:keyref>
        
        <xsd:key name="PersonKey">
            <xsd:selector xpath="Person" />
            <xsd:field xpath="@ID" />
        </xsd:key>
        <xsd:keyref name="PersonKeyRef_Crisis" refer="PersonKey">
            <xsd:selector xpath="Crisis/People/Person" />
            <xsd:field xpath="@ID" />
        </xsd:keyref>
        <xsd:keyref name="PersonKeyRef_Org" refer="PersonKey">
            <xsd:selector xpath="Organization/People/Person" />
            <xsd:field xpath="@ID" />
        </xsd:keyref>
        
        <xsd:key name="OrgKey">
            <xsd:selector xpath="Organization" />
            <xsd:field xpath="@ID" />
        </xsd:key>
        <xsd:keyref name="OrgKeyRef_Crisis" refer="OrgKey">
            <xsd:selector xpath="Crisis/Organizations/Org" />
            <xsd:field xpath="@ID" />
        </xsd:keyref>
        <xsd:keyref name="OrgKeyRef_Person" refer="OrgKey">
            <xsd:selector xpath="Person/Organizations/Org" />
            <xsd:field xpath="@ID" />
        </xsd:keyref>
    </xsd:element>
    
    <xsd:complexType name="CrisisType">
        <xsd:sequence>
            <xsd:element name="People" minOccurs="0" maxOccurs="1">
                <xsd:complexType>
                    <xsd:sequence>
                        <xsd:element name="Person" type="PersonWithID" minOccurs="1" maxOccurs="unbounded" />
                    </xsd:sequence>
                </xsd:complexType>
            </xsd:element>
            <xsd:element name="Organizations" minOccurs="0" maxOccurs="1">
                <xsd:complexType>
                    <xsd:sequence>
                        <xsd:element name="Org" type="OrgWithID" minOccurs="1" maxOccurs="unbounded" />
                    </xsd:sequence>
                </xsd:complexType>
            </xsd:element>
            
            <xsd:element name="Kind" type="xsd:token" minOccurs="0" maxOccurs="1" />
            <xsd:element name="Date" type="xsd:date" minOccurs="0" maxOccurs="1" />
            <xsd:element name="Time" type="xsd:time" minOccurs="0" maxOccurs="1" />
            <xsd:element name="Locations" type="ListType" minOccurs="0" maxOccurs="1" />
            <xsd:element name="HumanImpact" type="ListType" minOccurs="0" maxOccurs="1" />
            <xsd:element name="EconomicImpact" type="ListType" minOccurs="0" maxOccurs="1" />
            <xsd:element name="ResourcesNeeded" type="ListType" minOccurs="0" maxOccurs="1" />
            <xsd:element name="WaysToHelp" type="ListType" minOccurs="0" maxOccurs="1" />
            <xsd:element name="Common" type="CommonType" minOccurs="0" maxOccurs="1" />
            <xsd:any minOccurs="0" />
        </xsd:sequence>
        <xsd:attribute name="ID" type="CrisisIDType" use="required" />
        <xsd:attribute name="Name" type="xsd:token" use="required" />
    </xsd:complexType>

    <xsd:complexType name="PersonType">
        <xsd:sequence>
            <xsd:element name="Crises" minOccurs="0" maxOccurs="1">
                <xsd:complexType>
                    <xsd:sequence>
                        <xsd:element name="Crisis" type="CrisisWithID" minOccurs="1" maxOccurs="unbounded" />
                    </xsd:sequence>
                </xsd:complexType>
            </xsd:element>
            <xsd:element name="Organizations" minOccurs="0" maxOccurs="1">
                <xsd:complexType>
                    <xsd:sequence>
                        <xsd:element name="Org" type="OrgWithID" minOccurs="1" maxOccurs="unbounded" />
                    </xsd:sequence>
                </xsd:complexType>
            </xsd:element>
            
            <xsd:element name="Kind" type="xsd:token" minOccurs="0" maxOccurs="1" />
            <xsd:element name="Location" type="xsd:token" minOccurs="0" maxOccurs="1" />
            <xsd:element name="Common" type="CommonType" minOccurs="0" maxOccurs="1" />
            <xsd:any minOccurs="0" />
        </xsd:sequence>
        <xsd:attribute name="ID" type="PersonIDType" use="required" />
        <xsd:attribute name="Name" type="xsd:token" use="required" />
    </xsd:complexType>

    <xsd:complexType name="OrgType">
        <xsd:sequence>
            <xsd:element name="Crises" minOccurs="0" maxOccurs="1">
                <xsd:complexType>
                    <xsd:sequence>
                        <xsd:element name="Crisis" type="CrisisWithID" minOccurs="1" maxOccurs="unbounded" />
                    </xsd:sequence>
                </xsd:complexType>
            </xsd:element>
            <xsd:element name="People" minOccurs="0" maxOccurs="1">
                <xsd:complexType>
                    <xsd:sequence>
                        <xsd:element name="Person" type="PersonWithID" minOccurs="1" maxOccurs="unbounded" />
                    </xsd:sequence>
                </xsd:complexType>
            </xsd:element>
            
            <xsd:element name="Kind" type="xsd:token" minOccurs="0" maxOccurs="1" />
            <xsd:element name="Location" type="xsd:token" minOccurs="0" maxOccurs="1" />
            <xsd:element name="History" type="ListType" minOccurs="0" maxOccurs="1" />
            <xsd:element name="ContactInfo" type="ListType" minOccurs="0" maxOccurs="1" />
            <xsd:element name="Common" type="CommonType" minOccurs="0" maxOccurs="1" />
            <xsd:any minOccurs="0" />
        </xsd:sequence>
        <xsd:attribute name="ID" type="OrgIDType" use="required"/>
        <xsd:attribute name="Name" type="xsd:token" use="required" />
    </xsd:complexType>

    <xsd:complexType name="CommonType">
        <xsd:sequence>
            <xsd:element name="Citations" type="ListType" minOccurs="0" maxOccurs="1" />
            <xsd:element name="ExternalLinks" type="ListType" minOccurs="0" maxOccurs="1" />
            <xsd:element name="Images" type="ListType" minOccurs="0" maxOccurs="1" />
            <xsd:element name="Videos" type="ListType" minOccurs="0" maxOccurs="1" />
            <xsd:element name="Maps" type="ListType" minOccurs="0" maxOccurs="1" />
            <xsd:element name="Feeds" type="ListType" minOccurs="0" maxOccurs="1" />
            <xsd:element name="Summary" type="xsd:token" minOccurs="0" maxOccurs="1" />
        </xsd:sequence>
    </xsd:complexType>

    <xsd:complexType name="ListType">
        <xsd:sequence>
            <xsd:element name="li" minOccurs="1" maxOccurs="unbounded">
                <xsd:complexType mixed="true">
                    <xsd:attribute name="href" type="xsd:token" />
                    <xsd:attribute name="embed" type="xsd:token" />
                    <xsd:attribute name="text" type="xsd:token" />
                </xsd:complexType>
            </xsd:element>
        </xsd:sequence>
    </xsd:complexType>

    <xsd:complexType name="CrisisWithID">
        <xsd:attribute name="ID" type="CrisisIDType" use="required" />
    </xsd:complexType>
    
    <xsd:complexType name="PersonWithID">
        <xsd:attribute name="ID" type="PersonIDType" use="required" />
    </xsd:complexType>
    
    <xsd:complexType name="OrgWithID">
        <xsd:attribute name="ID" type="OrgIDType" use="required" />
    </xsd:complexType>

    <xsd:simpleType name="CrisisIDType">
        <xsd:restriction base="IDType">
            <xsd:pattern value="CRI_[A-Z]{6}" />
        </xsd:restriction>
    </xsd:simpleType>
    
    <xsd:simpleType name="PersonIDType">
        <xsd:restriction base="IDType">
            <xsd:pattern value="PER_[A-Z]{6}" />
        </xsd:restriction>
    </xsd:simpleType>
    
    <xsd:simpleType name="OrgIDType">
        <xsd:restriction base="IDType">
            <xsd:pattern value="ORG_[A-Z]{6}" />
        </xsd:restriction>
    </xsd:simpleType>

    <xsd:simpleType name="IDType">
        <xsd:restriction base="xsd:token">
            <xsd:length value="10" />
        </xsd:restriction>
    </xsd:simpleType>
</xsd:schema>
\end{lstlisting}
\footnotesize
Figure 1. The XML schema % note under the picture
\normalsize

\newpage
\section*{Implementation}

\hfill \newline
Most of the code, including the \emph{import} and \emph{export} scripts, are done in Python.
The website is delivered using Django and the site is designed using Twitter's Bootstrap.
The data is stored to a MySQL server on campus.
\newline

\subsection*{Data Model}
The core of our data model starts with the tables \emph{Crisis}, \emph{Organizations}, \emph{Person}.
These three tables all have a many to many relation between each other.
We have a \emph{Common} table to hold data these three models can all have.
\emph{Crisis}, emph{Organizations}, and \emph{Person} can have 0 or 1 \emph{Common} objects.
We have an abstract model, \emph{AbstractListType}.
The database will never write an \emph{AbstractListType}, but abstract types are useful when scripting.
The types \emph{CommonListType} and \emph{CrisisListType} both inherit from \emph{AbstractListType}.
\emph{CrisisListType} will hold data for \emph{Crisis} objects, and \emph{CommonListType} will hold data for \emph{Common}.

\subsection*{Import}
\emph{Import} is implemented as a Python script, in the root folder of the application.
It reads in an XML file as input, parse the information, and stores the applicable data into the database for future viewing.
The XML file must conform to the schema in Figure 1.
Otherwise, no data will be imported
\emph{Import} is password protected, and only site administrators may run this script.
Running import is done on the website.
There is a link in the navbar that reads \emph{Import}.
From this page, an administrator may select a file and upload the information.

\subsection*{Export}
\emph{Export} is implemented as a Python script, in the root folder of the application.
It takes the data stored in the database and writes into a new XML file conforming to the schema in Figure 1.
\emph{Export} does not require administrator access.
Any user may export the data in this database by clicking the \emph{Export} link from the navigation bar.

% Show picture of data model
\newpage
\includegraphics*[width=4in,height=7in]{dataModelDiagram.png}
\newline
\footnotesize
Figure 2. A diagram of our data model % note under the picture
\normalsize
\newpage

\newpage
\section*{Testing}
\subsection*{unittest}

\newpage
\section*{Other}
% Other
%Proof-read your report. Get another group to read it. Read it aloud.
%Create diagrams with captions.
%Create sections and subsections effectively.

\end{document}  