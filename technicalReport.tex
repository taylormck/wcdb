\documentclass[12pt]{report}

\usepackage[pdftex]{graphicx} % for pictures

\usepackage{color}
\usepackage{listings}
\usepackage[usenames,dvipsnames,svgnames,table]{xcolor}

\usepackage{marvosym} % Required for symbols in the colored box
\usepackage{ifsym} % Required for symbols in the colored box

\usepackage{hyperref}
\definecolor{linkcolor}{HTML}{506266} % Blue grey color for links
\hypersetup{colorlinks,breaklinks, urlcolor=linkcolor, linkcolor=linkcolor} % Set up links and colors

\definecolor{shade}{HTML}{F5DD9D} % Peach color for the contact information box
\definecolor{text1}{HTML}{2b2b2b} % Main document font color, off-black
\definecolor{headings}{HTML}{701112} % Dark red color for headings

\usepackage{fancyhdr}
% \pagestyle{fancy}
% \fancyhf{}

\usepackage{titlesec} % Allows creating custom \section's

\titleformat{\section}{\color{headings}
\scshape\Large\raggedright}{}{0em}{}[\color{black}\titlerule]
\titlespacing{\section}{0pt}{0pt}{5pt} % Spacing around titles

\title{Great Moonstone Oxen of the First and Forsaken Oceans}
\author{
    Taylor McKinney
    \and Matt Stringer
    \and Cesar Cantu
    \and Spencer Reynoso
    \and Ben Bowley-Bryant
    \and David Denton}

\begin{document}
\maketitle

\section*{Introduction}
\subsection*{What is the problem?}
It's often easy to forget the many crises occurring throughout the world.
This site is designed to gather the data on every crisis, and present in a simple format.
We gather details about victims, and information on organizations and people that want to help.

\subsection*{What are the use cases?}
With the increasing availability of telecommunication access and this site makes use of this to share news and help victims.
This site can be a good source for anyone looking to get informed and help. 
We can provide contact information for people and organizations involved in helping the victims of crises around the world.
People who want to help can look to our site for the best ways to get involved, whether it be donating to the right charities or volunteering at a local site.

\newpage
\section*{Design}
\subsection*{XML Schema}
Our class has been divided into separate groups for this project, but most groups have agreed upon a shared schema for XML.
The schema as of the latest version is printed here, but the latest version can be found with this link.
\newline
\href{https://github.com/aaronj1335/cs373-wcdb1-schema/blob/master/WorldCrises.xsd.xml}{Updated schema on Github}

\newpage
\section*{Implementation}

\subsection*{Source Code}
Most of the code, including the \emph{import} and \emph{export} scripts, are done in Python.
The website is delivered using Django and the site is designed using Twitter's Bootstrap.
The data is stored to a MySQL server on campus.
Most of the separate groups have already shared data.
Our database already holds data from all of the shared groups, as well as our own.

\subsection*{Folder Structure}
The model of the data structure can be seen below.

% Here's the model of the data structure
\hfill \newline
\lstset{
    language=bash,
    frame=shadowbox,
    rulesepcolor=\color{gray},
    xleftmargin=50pt,
    xrightmargin=50pt,
    framexleftmargin=0pt,
    tabsize=2, 
    basicstyle=\normalsize, 
    numbers=none, 
}
\begin{lstlisting}
wcdb/
    crises/
    scripts/
    static/
        css/
        html/
        img/
        js/
        templates/
        xml/
\end{lstlisting}
\footnotesize Figure 1. The Folder Layout \normalsize

The root folder \emph{must} be named \emph{wcdb}.
If downloading from \emph{Github}, it is necessary to rename the cloned folder to \emph{wcdb}.
Inside the \emph{wcdb} folder, there are three folders \emph{crises}, \emph{scripts}, and \emph{static}.
The \emph{crisis} folder contains the \emph{crisis} application and the standard \emph{Django} files,
including \emph{models.py}, which has details about the table structure, and \emph{tests.py}, which contains
most of our unit tests.
The \emph{scripts} folder contains the scripts \emph{importScript} and \emph{export}.
The \emph{importScript} is named to avoid conflicts with the \emph{Python} keyword.
Inside the \emph{static} folder, there are the folders \emph{css}, \emph{html}, \emph{img}, \emph{js}, \emph{templates}, and \emph{xml}.
The \emph{css} folder contains the CSS files associated with Twitter's Bootstrap.
The \emph{html} folder contains all of our content pages.
The file \emph{base.html} is similar to a template.
It contains most of the HTML data that can be seen on the site, with several blocks filled in with content from other pages.
The other pages, excluding \emph{export.html} and \emph{export.xml} all extend \emph{base.html}.
When Django is rendering a page, it will start by rendering data from the page it extends.
Since all of our pages extend \emph{base.html}, all of our pages use the html found there.
When Django is filling in content from \emph{base.html} and encouters a block,
it will fill in this block with data from the first page it was rendering.

\subsection*{Data Model}
The core of our data model starts with the tables \emph{Crisis}, \emph{Organizations}, \emph{Person}.
These three tables all have a many to many relation between each other.
We have a \emph{Common} table to hold data these three models can all have.
\emph{Crisis}, emph{Organizations}, and \emph{Person} can have 0 or 1 \emph{Common} objects.
We have an abstract model, \emph{AbstractListType}.
The database will never write an \emph{AbstractListType}, but abstract types are useful when scripting.
The types \emph{CommonListType} and \emph{CrisisListType} both inherit from \emph{AbstractListType}.
\emph{CrisisListType} will hold data for \emph{Crisis} objects, and \emph{CommonListType} will hold data for \emph{Common}.

\subsection*{Import}
\emph{Import} is implemented as a Python script, in the \emph{scripts} folder.
It is named \emph{importScript.py} to avoid naming conflits with the \emph{Python} keyword.
It reads in an XML file as input, parse the information, and stores the applicable data into the database for future viewing.
The XML file must conform to the schema in Figure 1.
Otherwise, no data will be imported
\emph{Import} is password protected, and only site administrators may run this script.
Running import is done on the website.
There is a link in the navbar that reads \emph{Import}.
From this page, an administrator may select a file and upload the information.

\subsection*{Export}
\emph{Export} is implemented as a Python script, in the \emph{scripts} folder.
It is named \emph{export.py}, though may likely be renamed during a later stage to match \emph{importScript.py}.
It takes the data stored in the database and writes into a new XML file conforming to the schema in Figure 1.
\emph{Export} does not require administrator access.
Any user may export the data in this database by clicking the \emph{Export} link from the navigation bar.

% Show picture of data model
\hfill \newline \newline
\includegraphics*[width=3in,height=5in]{dataModelDiagram.png}
\newline
\footnotesize
Figure 2. A diagram of our data model % note under the picture
\normalsize
\newpage

\newpage
\section*{Testing}
Testing is done using tools provided by \emph{Django} and \emph{unittest}.
To run the tests, run the command \textbf{python manage.py test}.
This command will run the unit tests using \emph{SQLite3}.
\emph{SQLite3} was used for faster unit testing.
\emph{MySQL} should be used for testing in later phases of development to ensure the code is runing properly on our production environment.
The ideal solution is to create two testing environments, the first of which runs unit tests on \emph{SQLite3} for performance purposes,
and another to run tests on \emph{MySQL} before pushing to ensure there are no differences between the two that could cause errors.

\subsection*{unittest}
\emph{Django} extends the standard unittesting framework, and provides several hundred tests against the back end.
In addition, we've added a number of unit tests to test our scripts \emph{importScript} and \emph{export}.

\end{document}  