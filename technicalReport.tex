\documentclass[12pt]{report}

\usepackage{graphicx} % for pictures
\usepackage{wrapfig} % for wrapping text around figures

\usepackage{color}
\usepackage{listings}
\usepackage[usenames,dvipsnames,svgnames,table]{xcolor}

\usepackage{marvosym} % Required for symbols in the colored box
\usepackage{ifsym} % Required for symbols in the colored box

\usepackage{hyperref}
\definecolor{linkcolor}{HTML}{506266} % Blue grey color for links
\hypersetup{colorlinks,breaklinks, urlcolor=linkcolor, linkcolor=linkcolor} % Set up links and colors

\definecolor{shade}{HTML}{F5DD9D} % Peach color for the contact information box
\definecolor{text1}{HTML}{2b2b2b} % Main document font color, off-black
\definecolor{headings}{HTML}{701112} % Dark red color for headings

\usepackage{fancyhdr}
% \pagestyle{fancy}
% \fancyhf{}

\usepackage{titlesec} % Allows creating custom \section's

\titleformat{\section}{\color{headings}
\scshape\Large\raggedright}{}{0em}{}[\color{black}\titlerule]
\titlespacing{\section}{0pt}{0pt}{5pt} % Spacing around titles

\title{Great Moonstone Oxen of the First and Forsaken Oceans}
\author{
    Taylor McKinney
    \and Matt Stringer
    \and Cesar Cantu
    \and Spencer Reynoso
    \and Ben Bowley-Bryant
    \and David Denton}

\begin{document}
\maketitle

\section*{Introduction}
\subsection*{What is the problem?}
\hfill


It's often easy to forget the many crises occurring throughout the world.
This site is designed to gather the data on every crisis, and present it in a simple format.
We gather details about victims, and information about organizations and people that want to help.

\subsection*{What are the use cases?}
\hfill


There is increasing availability of internet access worldwide and this site makes use of this to share news and help victims.
This site can be a good source for anyone looking to get informed and help. 
We can provide contact information for people and organizations involved in helping the victims of crises around the world.
People who want to help can look to our site for the best ways to get involved, whether it be donating to the right charities or volunteering at a local site.


\newpage
\section*{Implementation}
\hfill


\subsection*{Source Code}
\hfill


Most of the code, including the \emph{importScript} and \emph{export} scripts, are done in Python.
The website is delivered using Django and the site is designed using Twitter's Bootstrap.
The data is stored to a MySQL server on campus.
Most of the separate groups have already shared data,
and our database already holds data from all of the shared groups, as well as our own.\\

\subsection*{Folder Structure}
\hfill


The model of the data structure can be seen below.\\

% Here's the model of the data structure
\lstset{
    language=bash,
    frame=shadowbox,
    rulesepcolor=\color{gray},
    xleftmargin=50pt,
    xrightmargin=50pt,
    framexleftmargin=0pt,
    tabsize=2, 
    basicstyle=\normalsize, 
    numbers=none, 
}
\begin{lstlisting}
wcdb/
    crises/
    scripts/
    static/
        css/
        html/
        img/
        js/
        templates/
        xml/
\end{lstlisting}
\footnotesize Figure 1. The Folder Layout \normalsize
\hfill\\


The root folder \emph{must} be named \emph{wcdb}.
If downloading from Github, rename \emph{wcdb}.
Inside the \emph{wcdb} folder, there are three folders \emph{crises}, \emph{scripts}, and \emph{static}.\\


The \emph{crisis} folder contains the \emph{crisis} application and the standard Django files,
including \emph{models.py}, which has details about the table structure, and \emph{tests.py}, which contains
most of our unit tests.\\


The \emph{scripts} folder contains the scripts \emph{importScript} and \emph{export}.
The \emph{importScript} is named to avoid conflicts with the \emph{Python} keyword.\\


Inside the \emph{static} folder, there are the folders \emph{css}, \emph{html}, \emph{img}, \emph{js}, \emph{templates}, and \emph{xml}.
The \emph{css} folder contains the CSS files associated with Twitter's Bootstrap.\\


The \emph{js} folder holds javascript files from Twitter's Bootstrap.
We currently use this javascript to animate our navigation sidebar.
When you hover your mouse over one of the links, its row will highlight.
The \emph{templates} folder holds just a single HTML file for now that is not in use.
Later, this folder will be used to hold templated HTML data that will be used recurringly throughout the site.
The \emph{xml} folder currently holds several XML that were temporarily used during development.
The only important file is \emph{WorldCrisis.xsd.xml}.
This file is the schema we test our XML data against in our \emph{importScript} and \emph{export}.\\


The \emph{html} folder contains all of our content pages.
The file \emph{base.html} is similar to a template.
It contains most of the HTML data that can be seen on the site, with several blocks filled in with content from other pages.
The other pages, excluding \emph{export.html} and \emph{export.xml} all extend \emph{base.html}.
When Django is rendering a page, it will start by rendering data from the page it extends.
Since all of our pages extend \emph{base.html}, all of our pages use the html found there.
When Django is filling in content from \emph{base.html} and encouters a block,
it will fill in this block with data from the page it was first passed.
The \emph{crisis}, \emph{organization}, and \emph{person} pages are our static pages.
They hold information that will be filled into the template by Django.
The static pages are still being delivered by Django.\\

\newpage
\subsection*{Data Model}
\hfill


% Show picture of data model
\begin{wrapfigure}[22]{r}{0.48\textwidth}
\vspace{-20pt}
\begin{flushright}
\includegraphics*[width=0.5\textwidth,height=4in]{dataModelDiagram.png}
\newline
\footnotesize
Figure 2. A diagram of our data model % note under the picture
\end{flushright}
\end{wrapfigure}
\normalsize

The core of our data model starts with the tables \emph{Crisis}, \emph{Organizations}, \emph{Person}.
These three tables all have a many to many relation between each other.
We have a \emph{Common} table to hold data these three models can all have.
\emph{Crisis}, emph{Organizations}, and \emph{Person} can have 0 or 1 \emph{Common} objects.
We have an abstract model, \emph{AbstractListType}.
The database will never write an \emph{AbstractListType}, but abstract types are useful when scripting.
The types \emph{CommonListType} and \emph{CrisisListType} both inherit from \emph{AbstractListType}.
\emph{CrisisListType} will hold data for \emph{Crisis} objects, and \emph{CommonListType} will hold data for \emph{Common}.


\subsection*{Import}
\hfill


\emph{Import} is implemented as a Python script, in the \emph{scripts} folder.
It is named \emph{importScript.py} to avoid naming conflits with the \emph{Python} keyword.
It reads in an XML file as input, parse the information, and stores the applicable data into the database for future viewing.
The XML file must conform to the schema in Figure 1.
Otherwise, no data will be imported
\emph{Import} is password protected, and only site administrators may run this script.
Running import is done on the website.
There is a link in the navbar that reads \emph{Import}.
From this page, an administrator may select a file and upload the information.


\subsection*{Export}
\hfill


\emph{Export} is implemented as a Python script, in the \emph{scripts} folder.
It is named \emph{export.py}, though may likely be renamed during a later stage to match \emph{importScript.py}.
It takes the data stored in the database and writes into a new XML file conforming to the schema in Figure 1.
\emph{Export} does not require administrator access.
Any user may export the data in this database by clicking the \emph{Export} link from the navigation bar.


\newpage
\section*{Design}
\subsection*{XML Schema}
\hfill


Our class has been divided into separate groups for this project, but most groups have agreed upon a shared schema for XML.
The schema as of the latest version is printed here, but the latest version can be found with this link.\\


\href{https://github.com/aaronj1335/cs373-wcdb1-schema/blob/master/WorldCrises.xsd.xml}{Updated schema on Github}\\


The entire XML file is wrapped in a \emph{WorldCrisis} object.
Within the WorldCrisis object there will be one or many \emph{Crisis}, \emph{Organization}, and \emph{Person} elements.
Each of those elements have uniquely typed key types, named \emph{CrisisKey}, \emph{OrgKey}, and \emph{PersonKey}.
There are wrappers for containers of these objects that can contain a list,
i.e., a Crises container for multiple Crisis objects, etc.
There are also type definitions for each type, i.e.,
\emph{CrisisType}, \emph{PersonType}, \emph{OrgType}.\\


There is a \emph{CommonType} definition for data that all three of the previous types have in common,
such as links, images, and videos.
There is a \emph{ListType} definition that contains a list of XML tokens to facilitate objects with lists of data.

\newpage
\section*{Testing}
\hfill


Testing is done using tools provided by Django and unittest.
To run the tests, run the command \textbf{python manage.py test}.
This command will run the unit tests using SQLite3.
SQLite3 was used for faster unit testing.
MySQL should be used for testing in later phases of development to ensure the code is runing properly on our production environment.
The ideal solution is to create two testing environments, the first of which runs unit tests on SQLite3 for performance purposes,
and another to run tests on MySQL before pushing to ensure there are no differences between the two that could cause errors.


Django extends the standard unittesting framework, and provides several hundred tests against the back end.
In addition, we've added a number of unit tests to test our scripts \emph{importScript} and \emph{export}.
There are plans to implement further testing for each view, but for now we are only showing static pages,
and the views are easily checked manually.


\newpage
\section*{Future Goals}
\hfill


Our primary future goal is to get the site presenting data from the database dynamically.
Our site is inspired by IMBD, and we will attemp to have similar functionality.
Some of these items we would like to have are:
\begin{itemize}
\item Each page should have a gallery for its related images
\item A search page, where users can search for any key and we return any relavant pages
\item Organize pages by filters, like date and location
\item A short list of recent events on the home page based on how kind they are
\item Improve the display of data, including charts and graphs
\end{itemize}

Some other features we would like to implement are:
\begin{itemize}
\item A  "Kindness Slider", where users can vote and rate organizations and people
\item A proper logo for our team and site
\item A random crisis action, to show people random crises
\item A ¨sympathy button¨ for crises
\item User profiles
\end{itemize}


\end{document}  